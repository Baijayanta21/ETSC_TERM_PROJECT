\documentclass[12pt,a4paper,twoside,symmetric]{tufte-handout}

% ===================== Packages =====================
\usepackage{graphicx}      % For images
\usepackage{hyperref}      % Clickable links
\usepackage{epigraph}      % Quotes
\usepackage{booktabs}      % Tables
\usepackage{amsmath,amssymb} % Math
\usepackage{parskip}       % Paragraph spacing
\usepackage{xcolor}        % Colors
\hypersetup{
    colorlinks=true,
    linkcolor=blue,
    urlcolor=cyan,
    citecolor=cyan,
}
\usepackage{listings}
\usepackage{xcolor}
\usepackage{inconsolata} % Optional: professional monospaced font
\usepackage{tcolorbox}
% Define custom Python style
\lstdefinestyle{pythonstyle}{
    language=Python,
    basicstyle=\ttfamily\small,          % Monospaced font for code
    keywordstyle=\color{blue}\bfseries,  % Keywords in blue bold
    stringstyle=\color{orange},          % Strings in orange
    commentstyle=\color{green!50!black}\itshape, % Comments in green italic
    numberstyle=\tiny\color{gray},       % Line numbers in gray, tiny font
    numbers=left,
    stepnumber=1,
    numbersep=8pt,
    backgroundcolor=\color{gray!8},      % Soft gray background
    frame=single,                        % Frame around code
    rulecolor=\color{gray!60},           % Frame color
    frameround=tttt,                      % Rounded corners
    tabsize=4,
    breaklines=true,                      % Wrap long lines
    showstringspaces=false,
    captionpos=b,                         % Caption below code
    xleftmargin=10pt,
    xrightmargin=10pt
}

% Shortcut to use it
\lstset{style=pythonstyle}
% Output style with line wrapping
\tcbset{outputbox/.style={
    colback=gray!10,
    colframe=black!50,
    boxrule=0.5pt,
    arc=4pt,
    left=4pt,
    right=4pt,
    top=2pt,
    bottom=2pt,
    fontupper=\ttfamily\itshape\small,  % change \small to \footnotesize, \scriptsize, \normalsize, \large, etc.
    enhanced,
    breakable,
    parbox=false
}}




\setlength{\textwidth}{15.2cm}         % main text width
\setlength{\marginparwidth}{2cm}     % margin (sidenotes) width
\setlength{\evensidemargin}{2cm}     % optional adjustment

\setlength{\topmargin}{-1.5cm}          % Space above header
\setlength{\headheight}{0cm}         % Header height
\setlength{\headsep}{1cm}          % Space between header and text
\setlength{\textheight}{27cm}        % Height of text area
\setlength{\footskip}{1.5cm}         % Space from bottom of text to footer


\usepackage{tikz}
\usetikzlibrary{shapes.geometric, arrows}

\tikzstyle{startstop} = [rectangle, rounded corners, minimum width=3cm, minimum height=1cm,text centered, draw=black, fill=red!30]
\tikzstyle{process} = [rectangle, minimum width=3cm, minimum height=1cm, text centered, draw=black, fill=blue!30]
\tikzstyle{decision} = [diamond, minimum width=3cm, minimum height=1cm, text centered, draw=black, fill=green!30]
\tikzstyle{arrow} = [thick,->,>=stealth]

\usepackage{ragged2e}  % Provides \justifying

% Enable numbered sections
\setcounter{secnumdepth}{3} % 1 = section, 2 = subsection, 3 = subsubsection


\usepackage{tabularx}
\usepackage{booktabs}
\usepackage{longtable}

% ===================== Metadata =====================
\title{Automated Student Report Generation from CSV to HTML via Python for Google Sites\\
CD61203:Essentials Tools for Scientific Computing}
\author{Baijayanta Bhattacharyya \\ 25PH91J10}
\date{\today}

\usepackage{tocloft}   
\renewcommand{\cftsecfont}{\normalsize\textsc}     
\renewcommand{\cftsubsecfont}{\normalsize\textsc}          
\renewcommand{\cftsecpagefont}{\color{blue}}       
\renewcommand{\cftdotsep}{2}                         
\setlength{\cftbeforesecskip}{0.5em}                
\setlength{\cftbeforesubsecskip}{0.25em}            


% Set font for List of Figures
\renewcommand{\cftloftitlefont}{\normalsize\bfseries}
\renewcommand{\cftfigfont}{\normalsize\textsc}          % figure entry font
\renewcommand{\cftfigpagefont}{\normalsize}      % page number font


% ===================== Document =====================
\begin{document}
\justifying
\maketitle

\begin{abstract}
{\Large\textsc{Abstract}}\\
\begin{abstract}
This report presents a complete workflow for \textbf{automating the generation of student reports} from CSV files into HTML format using Python, designed for easy embedding into Google Sites. The system reads structured student data, generates unique passcodes for secure access, and produces individual HTML reports that display each student’s marks, grades, and related details. Additionally, it incorporates an automated email distribution mechanism using Python’s SMTP library, ensuring that students receive their credentials securely. The report details the data structure, the Python implementation for data processing and passcode generation, the creation and styling of HTML reports, and practical examples of deployment. This approach streamlines the process of report preparation, enhances security, and allows scalability to handle large numbers of students efficiently.
\end{abstract}

\end{abstract}

\tableofcontents


% Add List of Listings right after ToC
% List of Listings (Code Snippets) with consistent font
\renewcommand{\lstlistlistingname}{Code Snippets} % Rename
\addcontentsline{toc}{section}{Code Snippets}    % Add to TOC
{\normalsize\textsc
\lstlistoflistings
}


\addcontentsline{toc}{section}{List of Figures} % add to TOC
{\normalsize\textsc{\listoffigures}}

\addcontentsline{toc}{section}{List of Tables} % add to TOC
{\normalsize\textsc{\listoftables}}
% Sections

\section{Introduction}

In academic institutions, generating individual student reports for courses is often repetitive and time-consuming. Traditionally, instructors manually create reports for each student, which is not only labor-intensive but also prone to errors.

This project aims to \textbf{automate the generation of student reports using Python}. The workflow reads student and course information from a CSV file, processes the data to compute or verify grades, generates unique passcodes for student authentication, and produces HTML files that are compatible with Google Sites for easy publication.

This automation ensures \textbf{accuracy, efficiency, and scalability}, allowing instructors to handle large numbers of students without manual intervention.


\section{Objectives}

The main objective of this project is to \textbf{automate the generation, distribution, and display of student reports} in a secure and scalable manner. The specific objectives include:

\begin{enumerate}
    \item \textbf{Automated Data Processing:} Read and process student and course information from a CSV file efficiently.
    \item \textbf{Secure Passcode Generation:} Generate unique and deterministic passcodes for each student to ensure secure access to their reports.
    \item \textbf{HTML Report Creation:} Produce individual HTML reports for each student, with proper formatting and styling, suitable for embedding into Google Sites.
    \item \textbf{Credential Distribution:} Send roll numbers and passcodes to students via automated email using Python’s SMTP functionality.
    \item \textbf{Scalability and Efficiency:} Ensure the system can handle large numbers of students without manual intervention while maintaining accuracy and consistency.
\end{enumerate}

\section{Methodology}

The methodology of the project is divided into the following stages:

\subsection{Data Input}
The project begins by reading a CSV file containing student details, including roll numbers, names, marks, and grades. Python processes the CSV data and stores it in a structured format (e.g., dictionaries or DataFrames) for further processing.

\subsection{Passcode Generation}
Each student is assigned a unique passcode for secure access to their report. In the current implementation, passcodes are \textbf{randomly generated} using Python's \texttt{random} and \texttt{string} modules. This ensures that each student receives a distinct alphanumeric code, which is included in the updated CSV file and used for authentication in the HTML report viewer.

The random generation approach is simple, effective, and ensures uniqueness across all students. In the near future, deterministic passcodes generated using a secret key and the student's roll number will be incorporated, allowing the administrator to regenerate passcodes as needed while maintaining security.


\subsection{CSV Update}
After passcode generation, a new column is added to the original CSV file to store each student's passcode. The updated CSV is saved, providing a complete record of all student credentials for administrative use.

\subsection{HTML Report Creation}
Python generates an HTML string for each student that acts as a secure report viewer. The HTML includes a login-like interface where students enter their roll number and passcode to access their individual report, which displays marks, grades, and other relevant details. The HTML is styled for clarity and compatibility with Google Sites.

\subsection{Credential Distribution}
Each student's roll number and passcode are automatically sent via email using Python's SMTP library. This ensures that students can securely access their reports without exposing credentials publicly.

\subsection{Deployment}
Finally, the generated HTML files are saved and embedded into Google Sites. Students can access their reports by entering their roll number and passcode, ensuring secure and personalized access.

\clearpage

\section{FLOWCHART}
\begin{figure}[h]
\centering
\begin{tikzpicture}[node distance=1.5cm, every node/.style={font=\small}, every arrow/.style={thick,->,>=stealth}]
% Styles
\tikzstyle{startstop} = [
    rectangle, 
    rounded corners=6pt, 
    minimum width=2.8cm, 
    minimum height=0.9cm, 
    text centered, 
    draw=black, 
    fill=red!15!orange!20, % soft coral
    inner sep=4pt, 
    font=\scshape,
    drop shadow
]

\tikzstyle{process} = [
    rectangle, 
    rounded corners=6pt, 
    minimum width=2.8cm, 
    minimum height=0.9cm, 
    text centered, 
    draw=black, 
    fill=blue!15!cyan!20, % soft sky blue
    inner sep=4pt, 
    font=\scshape,
    drop shadow
]

\tikzstyle{arrow} = [thick,->,>=stealth, draw=gray!80]


% Nodes
\node (start) [startstop] {Start};
\node (readxlx) [process, below of=start] {Convert Xlsx File to csv file};
\node (readcsv) [process, below of=readxlx] {Read CSV file with student data};
\node (generatepass) [process, below of=readcsv] {Generate random passcodes};
\node (updatecsv) [process, below of=generatepass] {Update CSV with passcodes};
\node (createhtml) [process, below of=updatecsv] {Generate HTML report viewer};
\node (sendemail) [process, below of=createhtml, align=center] 
{Send roll number \\ and passcode via email};
\node (deploy) [process, below of=sendemail] {Embed HTML into Google Sites};
\node (stop) [startstop, below of=deploy] {End};

% Arrows (curved for aesthetics)
\draw [arrow] (start) -- (readxlx);
\draw [arrow] (readxlx) -- (readcsv);
\draw [arrow] (readcsv) -- (generatepass);
\draw [arrow] (generatepass) -- (updatecsv);
\draw [arrow] (updatecsv) -- (createhtml);
\draw [arrow] (createhtml) -- (sendemail);
\draw [arrow] (sendemail) -- (deploy);
\draw [arrow] (deploy) -- (stop);

\end{tikzpicture}
\caption{Professional Compact Workflow of Automated Student Report Generation}
\label{fig:compact_workflow}
\end{figure}

\subsection{The CSV file structure}

\begin{table}[htbp]
\centering
\small
\renewcommand{\arraystretch}{1.5} % compact row height
\caption{ROLL LIST OF ETSC TERM PROJECT (CD61203, LTP : 3-0-0, CRD : 3) \\ Faculty: Baijayanta Bhattacharyya}
\label{tab:etsc_rolllist}

\resizebox{\textwidth}{!}{%
\begin{tabular}{c || l |l |c| c |c| c |c |c |l| l |l| l| l}
\toprule
S.No & Roll No & Name & Type & TA & MI & EN & TM & Grade & Remarks & Email & Institute Email & Mobile  \\
\midrule
1 & 25PH91J10 & Baijayanta Bhattacharyya & RS & 10 & 25 & 40 & 75 & A & .. & .. & .. & .. &  \\
2 & 25PH91J11 & Abhinaba Pahari & RS & 17 & 27 & 42 & 86 & EX & .. & .. & .. & .. &  \\
3 & 25PH91J25 & Sandipan Hazra & RS & 20 & 28 & 40 & 88 & EX & .. & .. & .. & .. &  \\
4 & 25PH91R05 & Anubhab Makhal & RS & 15 & 21 & 35 & 71 & B & .. & .. & .. & .. & \\
5 & 25PH91J03 & Bhaskar Mondal & RS & 18 & 22 & 37 & 77 & A & .. & .. & .. & .. &  \\
6 & 25PH91J27 & Soumyadiksha Halder & RS & 19 & 27 & 30 & 76 & A & .. & .. & .. & .. &  \\
7 & 25PH91J18 & Rajib Mahato & RS & 14 & 26 & 25 & 65 & B & .. & .. & .. & .. &  \\
\bottomrule
\end{tabular}%
}
\end{table}





\clearpage
\section{Code snippets and Results}

This section presents the workflow, implementation details, and examples of the generated reports. The discussion also highlights potential issues and future improvements.

\subsection{CSV Reading and Data Input}
At first we have a xlsx file which contains all the information. We save it as a CSV file. Then we start reading student data from a CSV file containing roll numbers, names, marks, and grades. Python's \texttt{csv} module is used to read the file and store the data in a structured format.

\begin{lstlisting}[caption={Importing Necessary Libraries and copy the data of csv file},captionpos=b]
# Import necessary Libraries
import random
import string
import csv
import numpy as np
import os
import matplotlib.pyplot as plt

# Absolute path of folder where the csv file is located
# Which contains all the necessary information
csv_file_path = "/Users/baijayantabhattacharyya/Downloads/ETSC_Proj/" 

# Name of the csv file
csv_file_name = "ETSC_Term_Project.csv"

# Combine folder path and file name safely into one valid file path.
csv_file = os.path.join(csv_file_path, csv_file_name) 

# Open the CSV file for reading
# The 'with' statement ensures the file closes automatically when done
with open(csv_file, newline='') as f:
    
    
    # Skip the first two lines before DictReader starts reading
    # It contains info about the faculty and the course details 
    
    # read the first line and store it in a string 
    # It contains information about the course name 
    first_line = ''.join([next(f)]) 
    
    
    # read the second line and store it in a string 
    # It contains information about the faculty name 
    second_line = ''.join([next(f)])

    
    reader = csv.DictReader(f)

    # Save the headers (field names)
    headers = reader.fieldnames 

    # Print the headers
    print("Column headers:", headers,"\n")
    
    # This line reads the entire file into memory
    data = [row for row in reader]  # list of dictionaries

print(f"Done reading the {csv_file} file.\n")
\end{lstlisting}

\begin{tcolorbox}[outputbox, title=Output]

Column headers: ['Serial No', 'Roll No', 'Name', 'Type', 'TA', 'MI', 'EN', 'TM', 'Grade', 'Remarks', 'Email', 'Institute Email', 'Mobile'] \\
Done reading the /Users/baijayantabhattacharyya/Downloads/ETSC\_Proj/ETSC\_Term\_Project.csv file.
\end{tcolorbox}

This ensures that all student records are available for subsequent processing.

\subsection{Extracting some details}

\begin{lstlisting}[caption={Extracting necessary details},captionpos=b]
# Extract the Course details that are on first line
# first line is '"ROLL LIST OF  ETSC TERM PROJECT (CD61203, LTP : 3-0-0, CRD : 3)",1,,,,,,,,,,,\r\n'

# Find starting and ending indices
start = first_line.find('ROLL LIST OF') + len('ROLL LIST OF')
end = first_line.find(')')

# Extract and clean
course_name = first_line[start:end+1].strip()

print(f"Course Name: {course_name}\n") # prints name of the course

# store more details in strings
# Split at '('
title, details = course_name.split('(')
title = title.strip()
details = details.strip(')')  # remove closing parenthesis

# Split details by comma
details_list = [d.strip() for d in details.split(',')]
course_code = details_list[0]
other_details = ', '.join(details_list[1:])

print(f"Course Name:  {title}\n")
print(f"Course Code:  {course_code}\n")
print(f"Details:  {other_details}\n")

# Extract the Faculty details that are on second line
# second line is 'FACULTY NAME :  Baijayanta Bhattacharyya,,,,,,,,,,,,\r\n'

# Find starting and ending indices
start = second_line.find('FACULTY NAME : ') + len('FACULTY NAME :')
end = second_line.find(',')

# Extract and clean
faculty_name = second_line[start:end].strip()

print(f"Faculty Name: {faculty_name}\n") # prints name of the faculty

# print the number of students and course,faculty details
print(f"In this course named {course_name},\n\nTaught by {faculty_name},\n\nConsists a total of {len(data)} number of students.\n")

# Find the Average marks
print(f"The average marks is : {np.average([ float(row['TM']) for row in data]):.2f} out of 100.\n")

# Extract 'Grades' column values 
grades = [row['Grade'] for row in data if row['Grade']]

# Count occurrences of each grade
# Also sort grades by desired grade order

order = ['EX', 'A', 'B', 'C', 'D','F']   

grade_counts = {g: grades.count(g) for g in order if grades.count(g) != 0}

print(f"The grade count is : {grade_counts}\n")

# Alternative way to do the same
# from collections import Counter

# grade_counts = Counter(grades)
# grade_counts = {g: grade_counts_all[g] for g in order if grade_counts_all[g] != 0}

# print(grade_counts)
\end{lstlisting}

\begin{tcolorbox}[outputbox, title=Output]

Course Name: ETSC TERM PROJECT (CD61203, LTP : 3-0-0, CRD : 3)\\
Course Name:  ETSC TERM PROJECT\\
Course Code:  CD61203\\
Details:  LTP : 3-0-0, CRD : 3\\
Faculty Name: Baijayanta Bhattacharyya\\
In this course named ETSC TERM PROJECT (CD61203, LTP : 3-0-0, CRD : 3),\\
Taught by Baijayanta Bhattacharyya,\\
Consists a total of 7 number of students.\\
The average marks is : 76.86 out of 100.\\
The grade count is : {'EX': 1, 'A': 1, 'B': 1}\\
\end{tcolorbox}

\begin{lstlisting}[style=pythonstyle, caption={Plot the distribution}]
# Create a new figure window of specific size 
plt.figure(figsize=(5,3))

# Draw a bar chart
plt.bar(grade_counts.keys(), grade_counts.values(),color='skyblue', edgecolor='black')

# title of the plot
plt.title('Grade Distribution', fontsize=16, fontweight='bold', fontfamily='serif')

# Label the x-axis
plt.xlabel('Grades', fontsize=14, fontfamily='serif')

# Label the y-axis
plt.ylabel('Number of Students', fontsize=14, fontfamily='serif')

# Add grid lines on the y-axis for readability
plt.grid(axis='y', linestyle='--', alpha=0.5)

# Show count labels on top of bars 
for i, (g, v) in enumerate(grade_counts.items()):
    plt.text(i, v + 0.2, str(v), ha='center',fontsize=12,fontfamily='serif')

# Adjust layout automatically so labels/titles fit neatly
plt.tight_layout()

ymax = max(grade_counts.values()) # max y value
# Force y-axis ticks to be integers
plt.yticks(np.arange(0,2+ymax))

# set the ylimit
plt.ylim(0,2+ymax)

# save the plot
# create a new directory named outfigs
# if not exists and create and save it there

# Create 'outfigs' directory inside csv_file_path
outfigs_dir = os.path.join(csv_file_path, "outfigs")
os.makedirs(outfigs_dir, exist_ok=True)  # Create if not exists

# Build the image filename
img_filename = course_name.replace(" ", "_") + "_grades."

# Full path to save the image
img_file = os.path.join(outfigs_dir, img_filename)

print(f"Image will be saved at: {outfigs_dir} \n")

# png file
plt.savefig(img_file+"png", dpi=500,bbox_inches='tight') 

# pdf file
plt.savefig(img_file+"pdf", dpi=500,bbox_inches='tight')

# display the chart
plt.show()
\end{lstlisting}

\begin{tcolorbox}[outputbox, title=Output]
Image will be saved at: /Users/baijayantabhattacharyya/Downloads/ETSC\_Proj/outfigs
\end{tcolorbox}

\begin{figure}[h]
    \centering
    \includegraphics[width=0.5\linewidth]{ETSC_TERM_PROJECT_(CD61203,_LTP_:_3-0-0,_CRD_:_3)_grades.pdf}
    \caption{Grade Distribution}
    \label{fig:placeholder}
\end{figure}



\subsection{Passcode Generation}
Each student is assigned a unique random passcode for secure access to their report. In the current implementation, passcodes are generated using Python's \texttt{random} and \texttt{string} modules.

\begin{lstlisting}[caption={Generating random passcodes for students},captionpos=b]
# Generate unique passcodes for students

# Parameters
num_students = len(data)        # total number of students 
passcode_length = 8             # number of characters per passcode

# Allowed characters: uppercase and lowercsae letters + digits + few special characters
# Choose any of the above char sets
chars = string.ascii_letters + string.digits + "@#$%&"

# Use a set to ensure all passcodes are unique
passcodes = set()

# Keep generating until we have one code per student
while len(passcodes) < num_students:
    code = ''.join(random.choices(chars, k=passcode_length))
    
    # set automatically removes duplicates
    passcodes.add(code)         

# Convert the set to a list 
passcodes = list(passcodes)

# summary
print(f"Generated {len(passcodes)} unique passcodes for {num_students} students.\n")

\end{lstlisting}
\begin{tcolorbox}[outputbox, title=Output]
Generated 7 unique passcodes for 7 students.
\end{tcolorbox}

These passcodes are stored in the updated CSV file and later used for authentication in the HTML report. In the near future, deterministic passcodes using a secret key and roll number will be incorporated for easier administrative management.

\subsection{CSV Update}
After generating passcodes, the CSV file is updated with an additional column for storing each student's passcode. The updated CSV provides a complete record of all student credentials.

\begin{lstlisting}[style=pythonstyle, caption={Updating the csv file}]
# create a new csv file 
# same as old csv file adding just a column name Passcode

# Add new column with passcode
for ii,row in enumerate(data,start=0):
    row["passcode"] = passcodes[ii]

# Save updated data to CSV
# create a new directory named outcsv
# if not exists and create and save it there

# Create 'outcsv' directory inside csv_file_path
outcsv_dir = os.path.join(csv_file_path, "outcsv")
os.makedirs(outcsv_dir, exist_ok=True)  # Create if not exists

csv_name,csv_ext = os.path.splitext(csv_file_name)
updated_csv_filename = f"{csv_name}_passcode_updated{csv_ext}"
# Full path to save the csv file
updated_csv_file = os.path.join(outcsv_dir, updated_csv_filename)

print(f"Updated csv file name: {updated_csv_filename}\n")

print(f"Updated csv file will be saved at: {outcsv_dir} \n")


with open(updated_csv_file, "w", newline="") as f:
    # Write back first two lines
    f.write(first_line)
    f.write(second_line)
    writer = csv.DictWriter(f, fieldnames=data[0].keys())
    writer.writeheader()
    writer.writerows(data)
print(f"Updated csv file has been saved successfully.\n")
\end{lstlisting}
\begin{tcolorbox}[outputbox, title=Output]
Updated csv file name: ETSC\_Term\_Project\_passcode\_updated.csv\\
Updated csv file will be saved at: /Users/baijayantabhattacharyya/Downloads/ETSC\_Proj/outcsv \\
Updated csv file has been saved successfully.
\end{tcolorbox}


\subsection{HTML Report Creation}
Python generates an HTML string for each student that acts as a secure report viewer. The HTML includes a login-like interface where students enter their roll number and passcode to view their individual report. 

\begin{lstlisting}[style=pythonstyle, caption={Generating the string which contains all the passcodes details etc.}]
# html file format to get the values so that we
html_str = ""

# loop over all the students
for ii in range(len(data)):
    
    inner_str = "\""+data[ii]['Roll No'] +"\""+ "\
    "" : { passcode: \""+data[ii]['passcode']+ "\", "\
    "name: \"" + data[ii]['Name']+ "\","\
    "score: "+ (data[ii]['TM'])+ ","\
    "grade: \""+ (data[ii]['Grade']) + "\" },\n"
    
    html_str+=inner_str
    
# print the string
print(f"Generating the string that contains details about "\
     f"the students,roll number,marks,grades,passcode.\n")

print(f"{html_str}\n")
\end{lstlisting}
\begin{tcolorbox}[outputbox, title=Output]
Generating the string that contains details about the students,roll number,marks,grades,passcode.\\
"25PH91J10"     : { passcode: "Td\&m87AC", name: "Baijayanta Bhattacharyya",score: 75,grade: "A" },\\
"25PH91J25"     : { passcode: "qVNT8nO5", name: "Sandipan Hazra",score: 88,grade: "EX" },\\
"25PH91J11"     : { passcode: "9Tdlhd7g", name: "Abhinaba Pahari",score: 86,grade: "EX" },\\
"25PH91R05"     : { passcode: "WkIKps7W", name: "Anubhab Makhal ",score: 71,grade: "B" },\\
"25PH91J03"     : { passcode: "pYLbFNT2", name: "Bhaskar Mondal",score: 77,grade: "A" },\\
"25PH91J27"     : { passcode: "Yh\%1jcVk", name: "Soumyadiksha Halder",score: 76,grade: "A" },\\
"25PH91J18"     : { passcode: "qm8Gs#z6", name: "Rajib Mahato",score: 65,grade: "B" },
\end{tcolorbox}
\subsection*{HTML String formation}
\begin{lstlisting}[style=pythonstyle, caption={HTML string formation}]
html_content = f"""<!DOCTYPE html>
<html>
<head>
<meta charset="UTF-8">
<title>{course_name}, F.M.= 100</title>

<!-- Google Fonts -->
<link 
    href="https://fonts.googleapis.com/css2?family=Poppins:wght@400;500;600&family=Fira+Code:wght@400;500&display=swap" 
    rel="stylesheet"
>

<!-- jsPDF library -->
<script 
    src="https://cdnjs.cloudflare.com/ajax/libs/jspdf/2.5.1/jspdf.umd.min.js">
</script>


<style>
  body {{
    margin: 0;
    background: #f5f7fa;
    font-family: 'Poppins', sans-serif;
    text-align: center;
    color: #333;
  }}

  h2 {{
    font-size: 26px;
    font-weight: 700;
    background: linear-gradient(90deg, #2b6cb0, #4a90e2);
    -webkit-background-clip: text;
    -webkit-text-fill-color: transparent;
    margin-bottom: 10px;
    letter-spacing: 0.4px;
  }}

  p {{
    color: #444;
    font-size: 17px;
    font-weight: 500;
    margin-top: 0;
    margin-bottom: 20px;
    letter-spacing: 0.3px;
  }}

  input {{
    padding: 10px;
    margin: 5px;
    border: 1px solid #ccc;
    border-radius: 5px;
    font-size: 15px;
    width: 200px;
    transition: all 0.3s ease;
  }}
  input:focus {{
    border-color: #2b6cb0;
    box-shadow: 0 0 5px rgba(43,108,176,0.3);
    outline: none;
  }}

  button {{
    padding: 10px 20px;
    background: #2b6cb0;
    color: white;
    border: none;
    border-radius: 5px;
    cursor: pointer;
    font-size: 15px;
    font-weight: 500;
    transition: background 0.3s ease;
  }}
  button:hover {{
    background: #4a90e2;
  }}

  #output {{
    display: none;
    background: white;
    padding: 20px;
    margin-top: 25px;
    border-radius: 10px;
    box-shadow: 0 3px 8px rgba(0,0,0,0.1);
    max-width: 450px;
    margin-left: auto;
    margin-right: auto;
    text-align: left;
    animation: fadeIn 0.5s ease-in-out;
  }}

  #studentName {{
    text-align: center;
    font-size: 20px;
    font-weight: 600;
    letter-spacing: 0.4px;
    color: #2b6cb0;
    text-shadow: 0 1px 2px rgba(0,0,0,0.1);
    margin-bottom: 10px;
  }}

  #studentScore {{
    font-family: 'Fira Code', monospace;
    font-size: 16px;
    color: #333;
    line-height: 1.6;
    white-space: pre;
  }}

  #greetingBox {{
    display: none;
    background: #dbeafe;
    color: #1e40af;
    padding: 14px;
    margin-top: 15px;
    border-radius: 8px;
    font-size: 16px;
    font-weight: 500;
    text-align: center;
    box-shadow: 0 2px 6px rgba(0,0,0,0.1);
  }}

  #pdfBtn {{
    display: none;
    margin-top: 15px;
    padding: 8px 16px;
    background: #10b981;
    color: white;
    border: none;
    border-radius: 5px;
    cursor: pointer;
    font-size: 15px;
    font-weight: 500;
  }}

  footer {{
    margin-top: 50px;
    padding: 14px 0;
    background: #f5f7fa;
    color: #4b5563;
    font-size: 15px;
    text-align: center;
    font-weight: 500;
    border-top: 1px solid #e5e7eb;
    letter-spacing: 0.4px;
    font-family: 'Poppins', sans-serif;
    text-shadow: 0 1px 2px rgba(0,0,0,0.05);
  }}

  @keyframes fadeIn {{
    from {{ opacity: 0; transform: translateY(8px); }}
    to {{ opacity: 1; transform: translateY(0); }}
  }}
</style>
</head>

<body>
  <div style="padding:30px;">
    <h2>{course_name}, F.M.= 100</h2>

    <p>
      Please enter your <b>Roll Number</b> and <b>Passcode</b><br>
      (sent to your email and also to official <i>KGPian</i> mail ID) to view your marks.
    </p>

    <form id="resultForm" style="margin-top:20px;">
      <input type="text" id="roll" placeholder="Roll Number" required><br>
      <input type="password" id="passcode" placeholder="Passcode" required><br>
      <button type="submit">View Result</button>
    </form>

    <div id="output">
      <h3 id="studentName"></h3>
      <pre id="studentScore"></pre>
      <div id="greetingBox"></div>
      <button id="pdfBtn">Download PDF</button>
    </div>
  </div>

  <footer>
  <div>&copy; 2025 Baijayanta Bhattacharyya. All rights reserved.</div>
  <div>Contact: <a href="mailto:baijayantabhattacharyya2021@gmail.com"
  >baijayantabhattacharyya2021@gmail.com</a></div>

  <script>
    const studentData = {{
      {html_str}
    }};

    document.getElementById("resultForm").addEventListener("submit", function(e) {{
      e.preventDefault();

      const roll = document.getElementById("roll").value.trim();
      const pass = document.getElementById("passcode").value.trim();
      const output = document.getElementById("output");
      const nameEl = document.getElementById("studentName");
      const scoreEl = document.getElementById("studentScore");
      const greetingBox = document.getElementById("greetingBox");
      const pdfBtn = document.getElementById("pdfBtn");

      scoreEl.style.textAlign = "left";
      greetingBox.style.display = "none";
      pdfBtn.style.display = "none";

      if (!studentData[roll]) {{
        nameEl.textContent = "❌ Invalid Student Roll Number";
        scoreEl.textContent = "You haven’t registered in this course.";
        scoreEl.style.textAlign = "center";
        output.style.display = "block";
        return;
      }}

      const student = studentData[roll];

      if (student.passcode !== pass) {{
        nameEl.textContent = "⚠️ Wrong Password";
        scoreEl.textContent = "Enter again...";
        scoreEl.style.textAlign = "center";
        output.style.display = "block";
        return;
      }}

      nameEl.textContent = "✅ Password is correct";
      scoreEl.textContent =
`Student Name     : ${{student.name}}
Roll Number      : ${{roll}}
Marks Obtained   : ${{student.score}}/100
Grade Obtained   : ${{student.grade}}`;

      // Time-based greeting
      const now = new Date();
      const hour = now.getHours();
      let greetingText = "";
      if(hour >= 5 && hour < 12) {{
        greetingText = `�� Good Morning, ${{student.name}}`;
      }} else if(hour >= 12 && hour < 17) {{
        greetingText = `☀️ Good Afternoon, ${{student.name}}`;
      }} else if(hour >= 17 && hour < 21) {{
        greetingText = `�� Good Evening, ${{student.name}}`;
      }} else {{
        greetingText = `�� Good Night, ${{student.name}}`;
      }}

      greetingBox.textContent = greetingText;
      greetingBox.style.display = "block";
      pdfBtn.style.display = "inline-block";
      output.style.display = "block";

      pdfBtn.onclick = function() {{
    const {{ jsPDF }} = window.jspdf;
    const doc = new jsPDF();

    // --- Heading ---
    doc.setFont("Poppins", "italic");
    doc.setFontSize(25);
    doc.text("Results", 105, 20, {{ align: "center" }});

    // --- Horizontal line below heading ---
    doc.setLineWidth(0.5);
    doc.line(20, 25, 190, 25);

    // --- Starting Y position for student details ---
    let startY = 35;

    // Set line height factor (1.2 is slightly more spaced, default is ~1)
    doc.setLineHeightFactor(1.25);

    // --- Student & course details ---
    let text = 
        `Course Name       : {title}\n` +
        `Course Code       : {course_code}\n` +
        `Course Details    : {other_details}\n` +
        `Faculty Details   : {faculty_name}\n` +
        `Student Name      : ${{student.name}}\n` +
        `Roll Number       : ${{roll}}\n` +
        `Marks Obtained    : ${{student.score}}/100\n` +
        `Grade Obtained    : ${{student.grade}}`;

    doc.setFont("Courier", "normal");
    doc.setFontSize(15);
    doc.text(text, 20, startY);

    // --- Footer with current date and time ---
    const now = new Date();
    const dateTimeStr = now.toLocaleString();
    doc.setFont("times", "italic");
    doc.setFontSize(10);
    doc.text(`Generated on : ${{dateTimeStr}}`, 105, 280, {{ align: "center" }});
    
    // Save PDF
    doc.save(`Result_${{roll}}.pdf`);
}};
}});
</script>
</body>
</html>
"""
\end{lstlisting}

\begin{lstlisting}[style=pythonstyle, caption={Saving of HTML file}]
# html file name 
# course name + faculty_name 

html_file_name = course_name+"_"+faculty_name+"_results.html"

# replace all whitespaces with underscore

html_file_name = html_file_name.replace(" ","_")

print(f"The output HTML file name is : {html_file_name}\n")

# create a new directory named outhtml
# if not exists and create save it there

# Create 'outhtml' directory inside csv_file_path
outhtml_dir = os.path.join(csv_file_path, "outhtml")
os.makedirs(outhtml_dir, exist_ok=True)  # Create if not exists


# save the html file in the same directory as the csv file

print(f"Saving HTML file, saving location : {outhtml_dir}\n")

# Combine folder path and file name safely into one valid file path.

html_file = os.path.join(outhtml_dir,html_file_name)

# Save to file
with open(html_file, "w", encoding="utf-8") as f:
    f.write(html_content)

print(f"HTML file has been saved successfully.\n")
\end{lstlisting}
\begin{tcolorbox}[outputbox, title=Output]
The output HTML file name is : ETSC\_TERM\_PROJECT\_(CD61203,\_LTP\_:\_3-0-0,\_CRD\_:\_3)\\\_Baijayanta\_Bhattacharyya\_results.html\\
Saving HTML file, saving location : /Users/baijayantabhattacharyya/Downloads/ETSC\_Proj/outhtml\\
HTML file has been saved successfully.\\
\end{tcolorbox}
The HTML is styled for clarity and compatibility with Google Sites.

\subsection{Host the HTML to Google Sites}

Once the HTML report viewer is generated using Python, it needs to be made accessible to students. Google Sites provides a convenient platform to host these HTML files without requiring a separate web server. 

The workflow is as follows:

\begin{enumerate}
    \item Open your Google Site and navigate to the page where you want to embed the report.
    \item Click on the \textbf{Embed} option.
    \item Choose \textbf{Embed Code} and paste the HTML string or upload the HTML file generated by Python.
    \item Adjust the width and height of the embedded frame to properly display the report.
    \item Publish the site to make the report accessible to students.
\end{enumerate}

Using Google Sites ensures that students can securely access their individual reports by entering their \textsc{roll number} and \textsc{passcode} provided via email. This method avoids the need for complex server setup and allows instructors to easily update and republish reports.

\begin{lstlisting}[style=pythonstyle, caption={Link details}]
# copy and paste the html file content 
# and host it to a server or in my case google sites
# copy the page link and paste it here .

url_link = "https://sites.google.com/view/baijayantabhattacharyya03/teaching/etsc_term_project_2025"
\end{lstlisting}

\subsection{Credential Distribution}
Students receive their roll numbers and passcodes via automated email using Python's SMTP library. This ensures that only authorized students can access their reports.

\begin{lstlisting}[style=pythonstyle, caption={Sending email via python}]
# Send personalized emails to each student in the course.
# Each email contains instructions on how to view their class test results.
# The emails are sent automatically via Python using SMTP.
# The email body is HTML-formatted with Garamond font for a clean, professional look.
# Each student receives their own roll number and unique passcode to access their marks.

# import the necessary modules
import smtplib
from email.mime.text import MIMEText
from email.mime.multipart import MIMEMultipart
from datetime import datetime

# Sender's details 
your_email = "sender@gmail.com" # put your email

# Gmail app password if 2FA is enabled
app_password = "16 word key"  # put google authentication code

# Email subject 
email_subject = f"{course_name} Results, F.M.= 100"
print(f"Email Subject : {email_subject}\n")

# Send the email 
try:
    with smtplib.SMTP_SSL("smtp.gmail.com", 465) as server:
        #login at server
        server.login(your_email, app_password)

        # loop over all the recipients
        for ii in range(len(data)):

            # Recipient details 
            to_email = data[ii]['Email']
            
            # CC recipient
            cc_email = data[ii]['Institute Email']

            student_name = data[ii]['Name']
            roll_no = data[ii]['Roll No']
            passcode = data[ii]['passcode']

            # Get current date and time 
            current_time = datetime.now().strftime("%d-%m-%Y %H:%M:%S") 

            # HTML Email Body with monospace font
            email_body_html = f"""
            <html>
              <body style="font-family: 'Garamond', serif; line-height: 1.6; color: #333;">
                <p>Hello {student_name},</p>
            
                <p>This is an automated email sent via Python on {current_time}.</p>
            
                <p>You can view your marks by visiting the following link:<br>
                   <a href="{url_link}" target="_blank">{url_link}</a>
                </p>

                <p>To access your marks, <br>
                   please enter your roll number: {roll_no}<br>
                   and your unique passcode: {passcode}</p>
            
                <p>Wishing you all the very best for your results!</p>
            
                <p>Warm regards,<br>
                   {faculty_name}<br>
                   {title}, Course Instructor
                </p>
              </body>
            </html>
            """
            # Create email message 

            # Create a msg class object
            msg = MIMEMultipart()
            
            # Sender's email
            msg["From"] = your_email
            
            # receiver's email
            msg["To"] = to_email
            
            # Visible CC recipients
            msg["Cc"] = cc_email  
            
            # email subject
            msg["Subject"] = email_subject
            
            # attach email body
            msg.attach(MIMEText(email_body_html, "html"))
            
            # Sendmail takes a list of recipients, including CC
            server.sendmail(your_email, [to_email, cc_email], msg.as_string())

            # print sent message
            print(f"{ii+1}) Time : {current_time} , Email sent to {to_email} , CC: {cc_email}\n")
        
except Exception as e:
    print("Error:", e)
\end{lstlisting}
\begin{tcolorbox}[outputbox, title=Output]
Email Subject : ETSC TERM PROJECT (CD61203, LTP : 3-0-0, CRD : 3) Results, F.M.= 100\\

1) Time : 09-11-2025 17:12:16 ,\\
 Email sent to baijayantabhattacharyya2021@gmail.com ,\\
 CC: BAIJAYANTA21COSMO25@kgpian.iitkgp.ac.in\\

2) Time : 09-11-2025 17:12:16 ,\\
 Email sent to hazrasandipan55@gmail.com ,\\
 CC: SANDIPANHAZRA25@kgpian.iitkgp.ac.in\\

3) Time : 09-11-2025 17:12:16 ,\\
 Email sent to abhinabapahari@gmail.com ,\\
 CC: ABHINABA25@kgpian.iitkgp.ac.in\\

4) Time : 09-11-2025 17:12:16 ,\\
 Email sent to anubhabmakhal02@gmail.com ,\\
 CC: ANUBHABMAKHAL00225@kgpian.iitkgp.ac.in\\

5) Time : 09-11-2025 17:12:16 ,\\
 Email sent to bhaskar712146@gmail.com ,\\
 CC: BHASKAR71214625@kgpian.iitkgp.ac.in\\

6) Time : 09-11-2025 17:12:16 ,\\
 Email sent to soumyadiksha@gmail.com ,\\
 CC: SOUMYADIKSHAH25@kgpian.iitkgp.ac.in\\

7) Time : 09-11-2025 17:12:16 ,\\
 Email sent to rmgpm2018@gmail.com ,\\
 CC: RMGPM25@kgpian.iitkgp.ac.in
\end{tcolorbox}

\subsection{HTML site Results}
Upon entering the right credentials it pops up a screen where all the details are present Figure~\ref{fig:success} and \ref{fig:TheInterfacePage}. 
You can also download this in a pdf file. If you enter a wrong password then It shows Try again \ref{fig:Wrong Password} and if you enter wrong roll number then it shows that you are not registered in this course \ref{fig:Wrong Roll Number}.

\begin{figure}
    \centering
    \includegraphics[width=0.7\linewidth]{success_25PH91J25.png}
    \caption{The Interface Page}
    \label{fig:success}
\end{figure}

\begin{figure}
    \centering
    % Set width and/or height
    \includegraphics[width=0.7\linewidth, height=0.3\textheight, keepaspectratio]{view_25PH91j25.png}
    \caption{Upon Successful Validation}
    \label{fig:TheInterfacePage}
\end{figure}

\begin{figure}[h]
    \centering
    \includegraphics[width=0.7\linewidth]{Wrong_25PH91J25.png}
    \caption{Wrong Password}
    \label{fig:Wrong Password}
\end{figure}
\begin{figure}[h]
    \centering
    \includegraphics[width=0.7\linewidth]{wrong_25PH91J26.png}
    \caption{Wrong Roll Number}
    \label{fig:Wrong Roll Number}
\end{figure}

\begin{figure}[h]
    \centering
    \includegraphics[width=0.7\linewidth]{pdf_25PH91J25.png}
    \caption{PDF file view}
    \label{fig:PDF file view}
\end{figure}

\section{Discussions}
The automated workflow successfully generates secure, individualized student reports compatible with Google Sites. Random passcodes provide unique access for each student, while deterministic passcodes can be incorporated in future iterations for recoverability. 

\subsection{Potential Improvements:}
\begin{itemize}
    \item Add charts or interactive tables to the HTML reports for better visualization.
    \item Automate the embedding of HTML into Google Sites.
    \item Use deterministic passcodes for easier administrative management and recovery.
\end{itemize}

Overall, the project demonstrates a scalable, secure, and efficient approach to generating and distributing student reports.

\section{Project Status}
The project has been successfully implemented, including all major components: reading student data from CSV, generating random passcodes, updating CSV files, creating an HTML report viewer, sending credentials via email, and embedding the HTML into Google Sites. Currently, the project is fully functional with sample data and can be easily adapted to larger datasets. Future enhancements may include deterministic passcode generation using encryption and improved user interface for the HTML reports.

\section{Online Resources Used to Complete the Project}
The following online resources were consulted to implement various parts of the project:

\begin{itemize}
    \item \textbf{Python Documentation:} \url{https://docs.python.org/3/} – For Python language reference, libraries, and examples.
    \item \textbf{pandas Documentation:} \url{https://pandas.pydata.org/docs/} – For handling CSV files and data manipulation.
    \item \textbf{Python smtplib module:} \url{https://docs.python.org/3/library/smtplib.html} – For sending emails programmatically.
    \item \textbf{Google Sites Help:} \url{https://support.google.com/sites} – For embedding HTML and managing pages on Google Sites.
    \item \textbf{Stack Overflow and Community Forums} – For troubleshooting and implementation tips.
    \item \textbf{TikZ \& LaTeX documentation:} \url{https://ctan.org/pkg/pgf} – For creating flowcharts and professional diagrams.
\end{itemize}

\section{Note from the Author}
This project was developed as part of the ETSC Term Project (CD61203). The primary goal was to automate the generation and distribution of student reports in a secure and efficient manner. The author acknowledges that while the current implementation works with random passcodes, future updates will focus on deterministic and encrypted passcodes for enhanced security.
\begin{enumerate}
    \item The whole project was written in Python and this document has been compiled via \LaTeX.
    \item The very early draft was written by the Author, but has been refined heavily by \textsc{Generative AI Tools like CHATGPT}.
    \item The things related to this project is make publicly available by the Author  on \textsc{GitHub} \url{https://github.com/Baijayanta21/ETSC_TERM_PROJECT}
    \item The plots have been generated via \textsc{TiKz} package.
\end{enumerate}

\section{Acknowledgments}
The author would like to express sincere gratitude to mentors, instructors, and colleagues for their guidance, encouragement, and support throughout this project. Special thanks to the course instructors for providing valuable feedback and suggestions during the development process.

\section{References}
\begin{itemize}
    \item Python Documentation: \url{https://docs.python.org/3/}
    \item pandas Documentation: \url{https://pandas.pydata.org/docs/}
    \item Python smtplib module: \url{https://docs.python.org/3/library/smtplib.html}
    \item Google Sites Help: \url{https://support.google.com/sites}
    \item TikZ \& PGF Manual: \url{https://ctan.org/pkg/pgf}
    \item Stack Overflow: \url{https://stackoverflow.com/}
\end{itemize}


\end{document}
